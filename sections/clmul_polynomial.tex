%=============================================================================
\section{CLMUL and Polynomial Hashes}
\label{sec:clmul}
%=============================================================================

These hashes use carry-less multiplication (CLMUL) or polynomial arithmetic
over finite fields to achieve theoretically grounded collision bounds.

\paragraph{Carry-less vs.\ standard multiplication.}
In standard multiplication, partial products are added with carry propagation.
In carry-less multiplication (polynomial multiplication over $\mathrm{GF}(2)$),
partial products are XORed without carries:

\begin{center}
\begin{tikzpicture}[
    >=Latex,
    bit/.style={minimum width=0.4cm, minimum height=0.4cm, font=\scriptsize},
    label/.style={font=\small}
]
% Standard multiplication example: 1011 × 1101
\begin{scope}[shift={(-3.5,0)}]
    \node[label] at (0.8,1.8) {\textbf{Standard (with carry)}};
    \node[bit] at (0,1) {1}; \node[bit] at (0.4,1) {0}; \node[bit] at (0.8,1) {1}; \node[bit] at (1.2,1) {1};
    \node[bit] at (-0.4,1) {$\times$};
    \node[bit] at (0,0.6) {1}; \node[bit] at (0.4,0.6) {1}; \node[bit] at (0.8,0.6) {0}; \node[bit] at (1.2,0.6) {1};
    \draw (-0.2,0.35) -- (1.4,0.35);
    \node[font=\scriptsize, right] at (1.6,0) {partial products};
    \node[font=\scriptsize, right] at (1.6,-0.4) {added with carry};
    \draw (-0.2,-0.65) -- (1.4,-0.65);
    \node[font=\scriptsize] at (0.6,-1) {$= 143$ (8 bits)};
\end{scope}

% Carry-less multiplication
\begin{scope}[shift={(3.5,0)}]
    \node[label] at (0.8,1.8) {\textbf{Carry-less (XOR)}};
    \node[bit] at (0,1) {1}; \node[bit] at (0.4,1) {0}; \node[bit] at (0.8,1) {1}; \node[bit] at (1.2,1) {1};
    \node[bit] at (-0.4,1) {$\otimes$};
    \node[bit] at (0,0.6) {1}; \node[bit] at (0.4,0.6) {1}; \node[bit] at (0.8,0.6) {0}; \node[bit] at (1.2,0.6) {1};
    \draw (-0.2,0.35) -- (1.4,0.35);
    \node[font=\scriptsize, right] at (1.6,0) {partial products};
    \node[font=\scriptsize, right] at (1.6,-0.4) {XORed (no carry)};
    \draw (-0.2,-0.65) -- (1.4,-0.65);
    \node[font=\scriptsize] at (0.6,-1) {$= 127$ (7 bits)};
\end{scope}
\end{tikzpicture}
\end{center}

\noindent
The PCLMULQDQ instruction performs 64$\times$64$\to$128 carry-less multiplication
in a single cycle on modern CPUs.

\paragraph{Pseudocode primitives.}
\label{sec:clmul-primitives}
\begin{align}
\Clmul(a, b) &= a \otimes b \quad\text{(128-bit carry-less product of two 64-bit inputs)}
\end{align}
Both map to a single \texttt{PCLMULQDQ} instruction with $\sim$1~cycle throughput.
The symbol $\otimes$ denotes polynomial multiplication in $\mathrm{GF}(2)[x]$ (XOR replaces addition, AND replaces multiplication of coefficients).

\subsection{CLHash}
\label{sec:clhash}

Source: \href{https://gitlab.com/fwojcik/smhasher3/-/blob/main/hashes/clhash.cpp}{\texttt{hashes/clhash.cpp}}~\cite{clhash}.

\paragraph{SMHasher3 test results.}
CLHash fails 186 of 250 tests, the worst result of any hash in this survey.
Failures span nearly every category: Avalanche, BIC, Sparse, Permutation, Text,
TwoBytes, PerlinNoise, Bitflip, and extensive seed-related tests.
Many seed-test failures are expected: CLHash is a keyed hash designed for a full random
key, and SMHasher3's seed-sensitivity tests probe a regime outside its design model.
The non-seed failures (Avalanche, Sparse, Text) indicate that the CLMUL-based
compression without a strong finalizer leaves detectable bias in the output.

A theoretically-grounded hash by Daniel Lemire using carry-less multiplication (CLMUL instruction,
x86-64 Haswell+). Based on almost-universal hashing over $\mathrm{GF}(2^{127})$ with collision
bound $\epsilon \approx n/2^{64}$. Uses 133 random 64-bit keys derived from seed.

\subsubsection{Mathematical Foundation}

CLHash computes a polynomial hash over $\mathrm{GF}(2^{127})$ with irreducible polynomial $p(x) = x^{127} + x + 1$. (Irreducibility over $\mathrm{GF}(2^{128})$ ensures the field structure.)
The core operation is carry-less multiplication:
\begin{equation}
a \otimes b = a \cdot b \mod p(x)
\end{equation}
where $\cdot$ denotes polynomial multiplication in $\mathrm{GF}(2)[x]$ (XOR instead of addition, AND instead of multiplication for coefficients).

\subsubsection{Key Primitives}

\paragraph{\texorpdfstring{Lazy mod $2^{127} + 2 + 1$:}{Lazy mod 2\string^127 + 2 + 1:}}
Given 254-bit product $(A_{\text{hi}}, A_{\text{lo}})$:
\begin{equation}
A_{\text{lo}} \oplus (A_{\text{hi}} \ll 1) \oplus (A_{\text{hi}} \ll 2)
\end{equation}

\paragraph{128-to-64 reduction:}
Uses the irreducible polynomial $p(x) = x^{64} + x^4 + x^3 + x + 1$ (notation: $(64, 4, 3, 1, 0)$).
The reduction is performed in two steps:

\textbf{Step 1: CLMUL reduction.}
Let $C = x^4 + x^3 + x + 1$ (the low-degree terms of $p(x)$). Since $x^{64} \equiv C \pmod{p(x)}$,
we have $A_{\text{hi}} \cdot x^{64} \equiv A_{\text{hi}} \cdot C$:
\begin{equation}
Q_2 \gets \Clmul(A_{\text{hi}}, C)
\end{equation}
This produces a result up to $64 + 4 = 68$ bits (since $A_{\text{hi}}$ is 64 bits and $C$ is degree 4).

\textbf{Step 2: Table lookup for final bits.}
The CLMUL result $Q_2$ may have bits in positions 64--67. These are reduced using
a precomputed 16-entry table where entry $i$ contains $i \cdot C \bmod 2^{64}$:
\begin{center}
\begin{tabular}{c cccccccc}
\toprule
$i$ & 0 & 1 & 2 & 3 & 4 & 5 & 6 & 7 \\
$i \cdot C$ & 0 & 27 & 54 & 45 & 108 & 119 & 90 & 65 \\
\midrule
$i$ & 8 & 9 & 10 & 11 & 12 & 13 & 14 & 15 \\
$i \cdot C$ & 216 & 195 & 238 & 245 & 180 & 175 & 130 & 153 \\
\bottomrule
\end{tabular}
\end{center}
The high 4 bits of $Q_2$ (bits 64--67) index into this table via \texttt{pshufb}:
\begin{equation}
Q_3 \gets \mathit{table}[(Q_2 \gg 64) \land \texttt{0xF}]
\end{equation}

\textbf{Final combination:}\phantomsection\label{fn:precomp-reduction64}
\begin{equation}
\textsc{precompReduction64}(A) = A_{\text{lo}} \oplus Q_2 \oplus Q_3
\end{equation}
The table lookup replaces what would otherwise require another CLMUL for the
overflow bits, saving one instruction in the critical path.

\subsubsection{CLHash Algorithm}

\begin{algorithmic}[1]
\State Generate 133 random 64-bit key words $s_0, \ldots, s_{132}$ from seed using xorshift128+
\State $s_{\mathrm{poly}} \gets (s_{128} \mid s_{129} \ll 64) \land (\texttt{0x3FFFFFFF\,FFFFFFFF\,FFFFFFFF\,FFFFFFFF})$ \Comment{128-bit; top 2 bits cleared for $\mathrm{GF}(2^{127})$}
\State $s_{\mathrm{final}} \gets (s_{130}, s_{131}) \in \ut{128}$; \quad $s_{\mathrm{len}} \gets s_{132} \in \ut{64}$
\State $v \gets 0$ (128-bit)
\For{each block $(x_i)_{i=0}^{127} \in \ut{64}^{128}$} \Comment{1024 bytes; Horner: $v \gets v \otimes s_{\mathrm{poly}} \oplus h_i$}
    \If{not first block}
        \State $v \gets v \otimes s_{\mathrm{poly}}$
    \EndIf
    \For{$i \in \{0, \ldots, 63\}$} \Comment{64 word pairs}
        \State $v \gets v \oplus \Clmul((x_{2i} \oplus s_{2i}), (x_{2i+1} \oplus s_{2i+1}))$
    \EndFor
\EndFor
\For{remaining word pairs $(x_{2i}, x_{2i+1})$} \Comment{Tail: same CLMUL pattern}
    \State $v \gets v \oplus \Clmul((x_{2i} \oplus s_{2i}), (x_{2i+1} \oplus s_{2i+1}))$
\EndFor
\State $t \gets v \oplus s_{\mathrm{final}}$; \quad $v \gets \Clmul(\textsc{Lo}(t),\, \textsc{Hi}(t))$
\State $v \gets v \oplus \Clmul(s_{\mathrm{len}}, n)$ \Comment{Mix in length}
\State \Return $\hyperref[fn:precomp-reduction64]{\textsc{precompReduction64}}(v)$ \Comment{bitmix variant applies MurmurHash3 $\Fmix$ after}
\end{algorithmic}

\subsection{UMASH}
\label{sec:umash}

Source: \href{https://gitlab.com/fwojcik/smhasher3/-/blob/main/hashes/umash.cpp}{\texttt{hashes/umash.cpp}}~\cite{umash}.

\paragraph{SMHasher3 test results.}
UMASH-128 fails 128 of 250 tests.
Like CLHash, UMASH is a keyed hash expecting a full random key (generated via Salsa20),
and most failures are in seed-related tests (Seed, SeedSparse, SeedAvalanche, SeedBIC,
SeedBitflip, SeedBlockLen, SeedBlockOffset).
Non-seed failures include BIC, some Sparse configurations, Permutation (single-bit keys),
TwoBytes, PerlinNoise, and Bitflip---reflecting the inherent linearity of
CLMUL-based compression.

UMASH by Paul Khuong combines CLMUL-based compression with polynomial hashing
over $\mathrm{GF}(2^{61}-1)$. It produces two independent hash functions for
fingerprinting (128-bit output).

\subsubsection{Mathematical Foundation}

UMASH works modulo the Mersenne prime $p = 2^{61} - 1$. Key operations:

\paragraph{Modular addition (fast):}
\begin{equation}
\textsc{add\_mod\_fast}(x, y) = \begin{cases}
x + y + 8 & \text{if } x + y \text{ overflows} \\
x + y & \text{otherwise}
\end{cases}
\end{equation}

\paragraph{Modular multiplication:}
\begin{equation}
\textsc{mul\_mod\_fast}(m, x) = \textsc{add\_mod\_fast}(\textsc{Lo}(m \times x), 8 \cdot \textsc{Hi}(m \times x))
\end{equation}

\paragraph{Horner double update:}\phantomsection\label{fn:horner-double-update}
\begin{equation}
\begin{aligned}
\textsc{horner\_double\_update}(v, m_0, m_1, x, y) &=
\textsc{mul\_mod\_fast}(m_0, v + x) \\
&\quad + \textsc{mul\_mod\_fast}(m_1, y)
\end{aligned}
\end{equation}

\subsubsection{OH Compression}

The ``OH'' (One-pass Hash) compresses 256-byte blocks using CLMUL:

\phantomsection\label{fn:oh-compress}%
\begin{algorithmic}[1]
\State Interpret the 256-byte input as $(x_i)_{i=0}^{31} \in \ut{64}^{32}$
\State $v \gets 0$ (128-bit)
\For{$i \in \{0, \ldots, 14\}$} \Comment{15 CLMUL pairs}
    \State $v \gets v \oplus \Clmul(x_{2i} \oplus s_{2i}, x_{2i+1} \oplus s_{2i+1})$
\EndFor
\State \textbf{ENH finalization:}
\State $(t_0, t_1) \gets (x_{30} + s_{30},\, x_{31} + s_{31})$ \Comment{Final ENH chunk}
\State $t_2 \gets \textsc{Hi}(t_0 \times t_1) + \mathit{seed}$; $t_3 \gets \textsc{Lo}(t_0 \times t_1)$
\State \Return $(v_{\text{lo}} \oplus t_3,\, v_{\text{hi}} \oplus t_2 \oplus t_3)$
\end{algorithmic}

\subsubsection{UMASH Algorithm}

\noindent We focus on the main (long-key) structure. Short-key special casing and exact tail handling are summarized.

\begin{algorithmic}[1]
\State Generate 34 random 64-bit key words $s_0, \ldots, s_{33}$ from seed via Salsa20
\State Prepare polynomial multipliers $f, f^2$ in $\mathrm{GF}(2^{61}-1)$
\State $v \gets 0$
\For{each block $(x_i)_{i=0}^{31} \in \ut{64}^{32}$} \Comment{Main loop (256 bytes)}
    \State $(t_0, t_1) \gets \hyperref[fn:oh-compress]{\textsc{OH\_compress}}((x_i)_{i=0}^{31})$
    \State $v \gets \hyperref[fn:horner-double-update]{\textsc{horner\_double\_update}}(v, f^2, f, t_0, t_1)$
\EndFor
\State Process remaining bytes with variable-length OH (tail summarized)
\State \Return $(v \oplus \Rot{8}{v}) \oplus \Rot{33}{v}$ \Comment{Finalization}
\end{algorithmic}

\subsubsection{Fingerprinting (128-bit)}

For fingerprinting, UMASH computes two independent hashes using:
\begin{itemize}
\item Different polynomial multipliers $(f_0, f_0^2)$ and $(f_1, f_1^2)$
\item A ``twisted'' version using an LRC (longitudinal redundancy check) XOR~\cite{umash-twist}
\item Both hashes use the same OH keys but different polynomial accumulators
\end{itemize}

\subsection{Polymur}
\label{sec:polymur}

Source: \href{https://gitlab.com/fwojcik/smhasher3/-/blob/main/hashes/polymur.cpp}{\texttt{hashes/polymur.cpp}}~\cite{polymur}.

Polymur by Orson Peters is a polynomial hash over $\mathrm{GF}(2^{61}-1)$. Uses 56-bit
chunks to avoid reduction within chunk processing. The polynomial multiplier $s_0$ is chosen as a generator of the
multiplicative group (via $s_0 = 37^e$ for random odd $e$ coprime to $2^{61}-2$), with
$s_0^7 < 2^{60} - 2^{56}$ ensuring efficient reduction.

\begin{algorithmic}[1]
\State Keys: $(s_0, s_1) \gets$ secret parameters derived from seed \Comment{$s_0$ multiplier, $s_1$ tweak}
\State $v \gets 0$
\If{$n \leq 7$}
		    \State \Return $s_1 + (s_0 + x_0)(s_0^2 + n) \mod p$
\ElsIf{$n \geq 50$}
		    \While{$n \geq 50$} \Comment{Main loop: 7$\times$56-bit chunks per iteration}
		        \State Let $(x_0, \ldots, x_6)$ be the next 7 words (each 56 bits, zero-extended)
		        \State $t_0 \gets (s_0 + x_0)(s_0^6 + x_1)$
		        \State $t_1 \gets (s_0^2 + x_2)(s_0^5 + x_3)$
	        \State $t_2 \gets (s_0^3 + x_4)(s_0^4 + x_5)$
	        \State $t_3 \gets (v + x_6) \cdot s_0^7$
	        \State $v \gets (t_0 + t_1 + t_2 + t_3) \mod p$
	    \EndWhile
		    \State Finalize with $v \cdot s_0^{14}$
\EndIf
\State Handle tail (8--49 bytes) with overlapping reads
\State $s_2 \gets \texttt{0xe9846af9b1a615d}$ \Comment{mx3 finalizer}
\State $v \gets \Xorshift{32}(v) \cdot s_2$
\State $v \gets \Xorshift{32}(v) \cdot s_2$
\State $v \gets \Xorshift{28}(v)$
\State \Return $v + s_1$
\end{algorithmic}

\subsection{poly-mersenne}
\label{sec:poly-mersenne}

Source: \href{https://gitlab.com/fwojcik/smhasher3/-/blob/main/hashes/poly_mersenne.cpp}{\texttt{hashes/poly\_mersenne.cpp}}~\cite{poly-mersenne}.

A simple polynomial hash over the Mersenne prime $p = 2^{61}-1$. Outputs 32 bits.
Keys from SplitMix64: multiplier $s_0 < 2^{60}$ and random values $s_1, \ldots, s_{K+1}$ for
optional $K$-independence.

\paragraph{Core primitive.}\phantomsection\label{fn:mult-combine61}
The multiply-accumulate uses \emph{lazy Mersenne reduction}: since $2^{61} \equiv 1 \pmod{p}$,
a 128-bit product can be reduced by folding the bits above position~61 back into the low part
with a shift and add---no division required.

\begin{algorithmic}[1]
\Function{mult\_combine61}{$v, m, x$} \Comment{$(v \cdot m + x) \bmod (2^{61}-1)$}
    \State $(\mathit{lo}, \mathit{hi}) \gets v \times m$ \Comment{Full 128-bit product}
    \State $\mathit{lo} \gets \mathit{lo} + x$ \Comment{Accumulate input word}
    \State $t \gets (\mathit{hi} \ll 3) \mid (\mathit{lo} \gg 61)$ \Comment{Bits $\geq 2^{61}$: fold back}
    \State $\mathit{lo} \gets \mathit{lo} \land (2^{61}-1)$
    \State \Return $\mathit{lo} + t$ \Comment{Lazy: may exceed $p$ by at most 1}
\EndFunction
\end{algorithmic}

\noindent The final result is fully reduced ($v \gets v - p$ if $v \geq p$) once, after all rounds.

\begin{algorithmic}[1]
\State $v \gets n$ \Comment{Initialize with length}
\For{each word $x \in \ut{32}$}
    \State $v \gets \hyperref[fn:mult-combine61]{\textsc{mult\_combine61}}(v, s_0, x)$
\EndFor
\If{$K > 0$} \Comment{Optional $K$-independence transform}
    \State $t \gets v$; $v \gets s_1$
    \For{$i \in \{2, \ldots, K+1\}$}: $v \gets \hyperref[fn:mult-combine61]{\textsc{mult\_combine61}}(v, t, s_i)$
    \EndFor
\EndIf
\State \Return $v \bmod 2^{32}$ \Comment{After final reduction if $v \geq p$}
\end{algorithmic}

\noindent Variants with higher $K$ provide $K$-wise independence (collision bound $(n/2^{61})^{K+1}$):

\begin{center}
\begin{tabular}{lcc}
\toprule
Variant & $K$ & Collision bound \\
\midrule
poly-mersenne-deg0 & 0 & $n / 2^{61}$ \\
poly-mersenne-deg1 & 1 & $(n / 2^{61})^2$ \\
poly-mersenne-deg2 & 2 & $(n / 2^{61})^3$ \\
poly-mersenne-deg3 & 3 & $(n / 2^{61})^4$ \\
poly-mersenne-deg4 & 4 & $(n / 2^{61})^5$ \\
\bottomrule
\end{tabular}
\end{center}

Higher-degree variants provide $K$-wise independence at the cost of additional
multiplications, improving theoretical collision bounds.
